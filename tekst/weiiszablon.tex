%%%%%%%%%%%%%%%%%%%%%%%%%%%%%%%%%%%%%%%%%%%%%%%%%%%%%%%%%%%%%%%%%
%%% %
%%% % weiiszablon.tex
%%% % The Faculty of Electrical and Computer Engineering
%%% % Rzeszow University Of Technology diploma thesis Template
%%% % Szablon pracy dyplomowej Wydziału Elektrotechniki 
%%% % i Informatyki PRz
%%% % June, 2015
%%%%%%%%%%%%%%%%%%%%%%%%%%%%%%%%%%%%%%%%%%%%%%%%%%%%%%%%%%%%%%%%%

\documentclass[12pt,twoside]{article}

\usepackage{weiiszablon}
\usepackage{polski}

\author{Rafał Ileczko}

% np. EF-123456, EN-654321, ...
\studentID{EF-144087}

\title{Rozszerzona rzeczywistość we wspomaganiu obsługi urządzeń}
\titleEN{Augumented Reality in support of device operating}


%%% wybierz rodzaj pracy wpisując jeden z poniższych numerów: ...
% 1 = inżynierska	% BSc
% 2 = magisterska	% MSc
% 3 = doktorska		% PhD
%%% na miejsce zera w linijce poniżej
\newcommand{\rodzajPracyNo}{2}


%%% promotor
\supervisor{dr inż. Mariusz Oszust}
%% przykład: dr hab. inż. Józef Nowak, prof. PRz

%%% promotor ze stopniami naukowymi po angielsku
\supervisorEN{PhD Mariusz Oszust}

\abstract{Treść streszczenia po polsku}
\abstractEN{Treść streszczenia po angielsku}

\begin{document}

% strona tytułowa
\maketitle

\blankpage

% spis treści
\tableofcontents

\clearpage
\blankpage


\section*{Wykaz symboli, oznaczeń i skrótów (opcjonalny)}
\addcontentsline{toc}{section}{Wykaz symboli, oznaczeń i skrótów (opcjonalny)}%
% 1 $\div$ 2 stron wykaz ważniejszych symboli i oznaczeń (jeśli jest potrzebny).
\clearpage


\section{Wstęp}

Panującym obecnie trendem jest automatyzacja - słowo występujące wielokrotnie na każdej konferencji - dotyczy to każdego rodzaju przemysłu i rynku usług. Jej efekty już od dawna odczuwamy także w życiu codziennym. Niczym dziwnym jest widok osoby mówiącej do telefonu, aby ten podał drogę do miejsca docelowego, wyszukał wykwintną restaurację, czy nawet opowiedział dowcip. Przedsiębiorstwa całego świata inwestują w działy automatyzacji produkcji, zakupując kolejne manipulatory, czy też rozwoju, próbujące wymyśleć nowe sposoby na wyelminiowania człowieka z procesu. W chwili obecnej jest on najsłabszym ogniwem. Maszyny, czy też oprogramowanie jest w stanie szybciej działać, wykonywać obliczenia będąc przy tym nieporównywalnie dokładniejszym. Jednocześnie urządzenia nie biorą urlopów, ich wydajność nie spada, nie wymagają motywacji, a także, co najważniejsze, mogą pracować 24h na dobę. Coraz częściej pojawiają się gniazda produkcyjne, czy też magazyny w pełni autonomiczne; wymagające jedynie konserwacji.

Taka sytuacja ma miejsce nie tylko na obszarach produkcyjnych. W niektórych zawodach specjalizowanych coraz częściej modele sztucznej inteligencji mają lepsze osiągi niż specjaliści z dużym doświadczeniem. Tak jest np. w przypadku diagnozy niektórych schorzeń na podstawie danych o pacjencie oraz statystykach chorób.[X] W Estonii uruchomiono pierwszy na świecie sąd, gdzie wyroki dotycze drobnych przestępstw wydaje model uczenia maszynowego. W chwili obecnej takie narzędzia służą jako doradcy dla ludzi. Natomiast poprzez rozwój działu sztucznej inteligencji na świecie, będą pełniły coraz większą rolę. Można więc przewidywać, że w przeciągu kilku lat wyprą oni także specjalistów.

Jeśli obecne trendy się nie zmienią, w nowoczesnych procesach produkcyjnych będzie popyt na dwa typy pracowników: inżynierów tworzących nowe rozwiązania, oraz konserwatorów obecnych. Tutaj pole do automatyzacji jest mniejsze. Nie stworzyliśmy jeszcze sztucznej inteligencji tworzącej koncepcje nowych maszyn w sposób pragmatyczny, tj. uwzględniającej podaną jej dokładnej specyfikacji. Podobnie ciężko zastąpić konserwatorów, którzy do pracy potrzebują zarówno zwinnych rąk, szeregu narzędzi jak i pomysłowości w diagnozie usterek itd. Mają jednak oni pewne zasadnicze ograniczenia. Posiadają jedynie dwie ręce, oraz parę oczu mogącej patrzeć się w jedną stronę. Projekt wykonany w ramach niniejszej pracy stara się w pewnym stopniu znieść te ograniczenia, aby przyszli pracownicy mając do wykonania pewną pracę mogli skupić się na swoim zadaniu znacznie redukcując liczbę obiektów potencjalnie go rozpraszających i ograniczających.

Odpowiednim rozwiązaniem powyższych problemów wydaje się być Rozszerzona Rzeczwistość (ang. Augumented Reality - AR). Metodologia ta polega na nałożeniu na obraz rzeczywisty widziany przez użytkownika elementów wygenerowanych przez program. Najczęściej spotykane zastosowania AR to filtry w aplikacjach typu Messenger czy Instagram. Pozwalają one wykonywać zdjęcia czy filmy ze zmodyfikowanym obrazem. Odpowiednie programy biorą pod uwagę obraz widoczny na kadrze, dzięki osoba na nim się znajdująca ma pogrubioną twarz, dodane psie bądź kocie uszy itp. Innym popularnym zastosowaniem są gry AR, gdzie najpopularniejszym przykładem jest Pokemon GO. Aplikacja wykorzystuje lokalizację użytkownika; będąc w miejscu gdzie występują pokemony można uruchomić aplikację, a na ekranie gracz zobaczy model 3D stworka, z którym można nawiązać interakcję - walczyć, bądź go złapać. AR znajduje zastosowanie również w innych dziedzinach - Ikea udostepniła aplikację, która pozwala, po zeskanowaniu płaszczyzny podłogi, zwizualizować jak będzie prezentował się dany produkt. Na potrzeby tej technologii powstały również specjalnie okulary, które pokazują wygenerowany obraz zaraz przed oczami użytkownika, zamiast na ekranie.

Okulary AR są narzędziem nadającym się idealnie do celów konserwacji. Poprzez nałożenie na widoczny, rzeczywisty, obraz dodatkowych elementów informacyjnych pozwala bez oderwania wzorku od obiektu zainteresowania wykonać wiele czynności. Przykładowo mając przed oczami wartość mierzoną z miernika, można testować obwody o wiele szybciej. Dzięki kamerze przebieg procesów można nagrywać, bądź współpracować zdalnie z inną osobą, gdzie obie widzą ten sam obraz. Przykłady można mnożyć, a dodatkową zaletą jest fakt, że nikt nic w ten sposób nie traci - można wciąż wykonywać wszystkie czynności w sposób tradycyjny.

Autor za wkład własny uważa:
- stworzenie oprogramowania pozwalające na spełnienie wizji opisanej w powyższym akapicie. Docelowo ma ono wspomagać obsługę i konserwację urządzeń dzięki metodom rozszerzonej rzeczywistości. Zakłada się, że dostępne maszyny, które mają być obsługiwane posiadają moduły umożliwiające połączenie się z lokalną siecią LAN, oraz mogą za jej pomocą wysyłać swoje parametry protokołem MQTT.
- stworzone na potrzeby demonstacji pracy proste urządzenie imitujące prawdziwe, dużo bardziej skomplikowane. A którego celem jest przesyłanie prostych wartości w postaci ciągu znaków.


\clearpage


\section{Tekst zasadniczy -- I}

\subsection {R/AR/VR}
W roku 2019 fotorealistyczna grafika komputerowa nie robi już na nikim większego wrażenia. Dzięki technikom Motion Capture komputerowo generowane postacie poruszają się zupełnie naturalnie. Specjalne algorytmy są w stanie symulować setki tysięcy włosów w czasie rzeczywistym \cite{hairworks}. Pierwsze gry wykorzystujące RayTracing w czasie rzeczywistym są już na rynku gier komputerowych \cite{raytracinggames}, a badania nad dedykowanym sprzętem już się rozpoczęły \cite{raytracinghardware}. Oznacza to, że obecna technologia zbliża się do granicy fotorealiznu, jaki jesteśmy w stanie osiągnąć. Naturalnym rozwiązaniem wydaje się więc wyjście ze środowiska płaskich ekranów na rzecz bardziej naturalnych doznań.

Pierwszym znaczącym krokiem w tym kierunku była publiczna zbiórka pieniędzy na serwisie Kickstarter w 2012 roku na projekt Oculus Rift. Twórcy zebrali w ten sposób prawie 2,5 miliona dolarów amerykańskich na rozwój. Efektem ubocznym tego wydarzenia była znaczna popularyzacja terminu Wirtualna Rzeczywistość (ang. Virtual Reality - VR) wsród ludzi na świecie. Obecnie, a więc 7 lat od pierwotnej zbiórki, do dyspozycji graczy dostępne jest 8 zestawów gogli VR, a liczba ta jeszcze się zwiększy {steamvr}. Wirtualna rzeczywistość jest technologią pozwalającą na prezentowaniu użytkownikowi sztucznej, generowanej rzeczywistości. Dzięki specjalnym goglom, kontrolerom oraz technologii ich śledzenia, interakcja z otoczeniem dokładnie imituje rzeczywistą. Oznacza to, że ruch głowy użytkownika wywołuje identyczny ruch kamery w symulacji. Takie rozwiązanie pozwala na dużo większą iterakcję użytkownika z otoczeniem niż jak ma to miejsce w symulacjach komputerowych wyświetlanych na typowym ekranie. W roku 2018 zyski firm z tytułu technologii VR przekroczyły 3.6 miliarda dolarów amerykański, co jest trzydziesto procentowym wzrostem względem roku poprzedniego. {vr2018}

Innym podejściem do tematu modyfikacji rzeczywistości jest rzeczywistość rozszerzona (ang. Augumented Reality). Technologia ta kompromisem pomiędzy rzeczywistością faktyczną, a wirtualną.






\subsection {Zastosowanie AR w przemyśle}
Konserwacja jest procesem periodycznym, zajmującym mniejwięcej tyle samo czasu, przy czym czas ten jest indywidualny dla każdej maszyny. Diagnostyka i naprawy występują rzadko, natomiast często są czasochłonne, wymagają wykonania szeregu testów i pomiarów, aby diagnoza była możliwa. Wymagają zwykle również szeregu dokumentów raportujących co zaszło, aby można było przeprowadzić stosowne analizy. Oba przypadki, choć skrajnie różne, mają cechę wspólną - poprzez ograniczenia ludzkie zajmują więcej czasu niż by mogły.

Najbardziej prymitywnym sposobem na usprawnienie pracy dzięki technologii jest zaopatrzenie pracownika w różnego rodzaju elektronarzędzia, tablet do sporządzania raportów i przeglądania instrukcji, czy krótkofalówke do komunikacji. Są to rzeczy na pewno ulepszające proces, natomiast pewne na pewne rzeczy nie mają wpływu - człowiek wciąż musi samodzielnie wypełnić dokumenty, a podczas pracy może mieć szereg wskaźników, które musi śledzić na bieżąco, bądź podążać za procedurami, które nie zawsze da się dokładnie zapamiętać.

Dzięki technikom AR można część tych czynności ułatwić bądź zniwelować. Raport z opracji może zostać wygenerowany automatycznie dzięki nagraniu, bądź danym pobranym w jej trakcie. Testowanie obwodów może się odbywać bez odrywania od nich wzroku, dzięki pokazaniu pracownikowi bezpośrednio przed oczami wskazania miernika, bądź schemat. Nauka obsługi może się odbywać dzięki przedstawieniu zainteresowanemu każdego elementu urządzenia, z dodatkowymi informacji po wybraniu. Dzięki takim rozwiązaniom wydajność oraz dokładność wzrośnie, ponieważ nie człowiek będzie odpowiadał za część procesu, a dużo wydajniejszy program. Nie może zostać pominięty również czynnik ludzki, czyli ludzka pomyłka. Jej częstość zależy od wielu czynników, takich jak stan psychiczny człowieka, zmęczenie, czas pracy, czy zbliżające się terminy. W przypadku programów jest zgoła odwrotnie. Dobrze napisany będzie działał stabilnie, nie popełniając błędów. Dlatego też wspomaganie pracy programami komputerowymi wydaje się być szczególnie pożądane w newralgicznych momentach łańucha produkcyjnego.


Rozszerzona rzeczywistość zakłada wykorzystanie kontekstu obrazu, który jest dostarczony do programu. Jednym z podejść do wykorzystania kontekstu jest stosowanie metod sztucznej inteligencji. Metody te służą do stworzenia modelu, który na podstawie dostarczonych danych jest w stanie nauczyć się wykonywać pewne zadanie. W przypadku problemu, który przedstawia poniższa praca, interesująca może być sekcja uczenia maszynowego, nazywana "uczeniem nadzorowanym". Metody te polegają na przekazaniu do modelu sztucznej inteligencji zbioru danych wraz z etykietami zawierającymi poprawną reakcję. Specjalne algorytmy wykonują swoje obliczenia na każdym obiekcie zbioru, a następnie sprawdzają, czy obliczenie jest poprawne. Następnie, w zależności od odpowiedzi, modyfikują swoje parametry, aby wynik pokrywał się z dostarczonymi etykietami. Podstawowe algorytmy uczenia nadzorowanego mają jednak pewną zasadniczą wadę - są prymitywne i dla celów wizji komputerowej nie były najlepszych rozwiązaniem. Następcą prostych metod uczenia maszynowego jest uczenie głębokie (ang. Deep Learning - DL). Te metody są już szeroko wykorzystywane w projektach wizji komputerowej. Dzięki faktowi, że mogą wydobywać z obrazu różne nieoczywiste cechy (kształty, kolory, wzory) zakres możliwości rośnie. Najpopularniejszym wykorzystaniem nauczonych modeli jest rozpoznawianie wielu klas obiektów na obrazie.

DL ma jednak kilka ważnych wad. Aby skutecznie nauczyć model wykonywać swoje zadanie, konieczny jest duży zestaw danych. Należy je odpowiednio przygotować, więc poza samymi zasobami, uczenie kosztuje również czas. Przyjmuje się, że dla każdej klasy obiektów, konieczne jest przygotowanie około 1000 zdjęć. Drugą wadą jest moc obliczeniowa, której zarówno do procesu uczenia, jak i późniejszego działania trzeba dużo. Ostatnim minusem tego rozwiązania jest nieelastyczność. Modele uczenia głębokiego są typu "czarna skrzynka", a więc zawierają parametry dla człowieka zupełnie niezrozumiałe i zakres działań, które można podjąć jest dosyć wąski - ogranicza się w zasadzie do douczenia modelu.

Innym podejściem do analizy kontekstowej obrazu jest "dopasowanie cech" (ang. feature matching). Polega ono na wydobyciu z obrazu specjalnych cech. Pobierając je z obrazu referencyjnego (obiektu, który chcemy wykryć) oraz obrazu z kamery (na którym jest poszukiwany obiekt), można je porównać i wyszukać czy i gdzie znajduje się obiekt referencyjny. Przewagą tej metody nad uczeniem głębokim jest fakt, że:
\begin{enumerate}[label=\alph*), leftmargin=1.25cm]
	\item wymaga mniej mocy obliczeniowej,
	\item nie występuje zjawisko uczenia,
	\item dzięki możliwości dostosowania parametrów wykrywania cech, elastyczność rozwiązania jest większa.
\end{enumerate}

Feature matching jest metodą wykorzystywaną nie tylko do rozpoznawiania obiektów na obrazie. Dzięki temu, że część wykrytych cech będzie się pokrywać na dwóch obrazach mających wspólny obszar, możliwe jest łączenie zdjęć w panoramy. Inne popularne zastosowanie do stabilizacja obrazu. Więdzac, że kamera drga w osiach prostopadłych do osi obiektywnu, obraz końcowy można delikatnie przesuwać, dzięki czemu będzie on stabilny.


\subsection{Co dopisać}
\begin{enumerate}[label=\alph*), leftmargin=1.25cm]
\item Napisać więcej o feature matching, rozwinąć zasadę działania
\item Algorytmy wydobywające punkty kluczowe i deskryptory, jakie są i czym się różnią
\item Matchery, jakie są i czym się różnią
\item Pinhole camera model, wzory, obrazki
\item Homografia, wzory, obrazki
\item Jak zdążę zaimplementować, to filtr Kalmana
\item Arduino
\item O MQTT
\item python i C?
\end{enumerate}

\clearpage	

\section{Tekst zasadniczy -- II}

\subsection{Do napisania w tym rozdziale}
\begin{enumerate}[label=\alph*), leftmargin=1.25cm]
\item Przyjęte założenia, dlaczego na komputerze
\item Ogólne architektura docelowa i do celów demonstracyjnych, opis sposobu komunikacji
\item Opis rozwiązania po stronie algorytmu 
\item Porównanie działania róznych algorytmów (ORB/SILF/SURF), matcherów itd. i dlaczego wybrałem co wybrałem.
\item Opis rozwiązania po stronie arduino
\item Opis demo
\end{enumerate}

\clearpage

\section{Podsumowanie i wnioski końcowe}
\begin{enumerate}[label=\alph*), leftmargin=1.25cm]
\item Podsumowanie
\item Co działa słabo
\item Jak można to potencjalnie poprawić
\item Świetlana przyszłość AR
\end{enumerate}

\clearpage

\section*{Załączniki}
\addcontentsline{toc}{section}{Załączniki}

Według potrzeb zawarte i uporządkowane uzupełnienie pracy o dowolny materiał źródłowy (wydruk programu komputerowego, dokumentacja kons\-truk\-cyj\-no-\-tech\-no\-lo\-gicz\-na, konstrukcja modelu -- makiety -- urządzenia, instrukcja obsługi urządzenia lub stanowiska laboratoryjnego, zestawienie wyników pomiarów i obliczeń, informacyjne materiały katalogowe itp.).


\clearpage

\addcontentsline{toc}{section}{Literatura}

\begin{thebibliography}{11}
\bibitem{hairworks} https://www.nvidia.pl/object/nvidia-hairworks-pl.html. Dostęp 15.06.2019.
\bibitem{raytracinggames} https://news.developer.nvidia.com/real-time-path-tracing-and-denoising-in-quake-ii-rtx/. Dostęp 15.06.2019.
\bibitem{raytracinghardware} https://graphics.stanford.edu/papers/rtongfx/rtongfx.pdf. Dostęp 15.05.2019.
\bibitem{steamvr} https://store.steampowered.com/steamvr?l=polish. Dostęp 15.05.2019. 
\bibitem{vr2018} https://www.viar360.com/virtual-reality-market-size-2018/. Dostęp 15.05.2019. 

\end{thebibliography}

\clearpage

\makesummary

\end{document} 
